%%%%%%%%%%%%%%%%%
% This is an sample CV template created using altacv.cls
% (v1.7, 9 August 2023) written by LianTze Lim (liantze@gmail.com). Compiles with pdfLaTeX, XeLaTeX and LuaLaTeX.
%
%% It may be distributed and/or modified under the
%% conditions of the LaTeX Project Public License, either version 1.3
%% of this license or (at your option) any later version.
%% The latest version of this license is in
%%    http://www.latex-project.org/lppl.txt
%% and version 1.3 or later is part of all distributions of LaTeX
%% version 2003/12/01 or later.
%%%%%%%%%%%%%%%%

%% Use the "normalphoto" option if you want a normal photo instead of cropped to a circle
% \documentclass[10pt,a4paper,normalphoto]{altacv}

\documentclass[10pt,letterpaper,withhyper]{altacv}
%% AltaCV uses the fontawesome5 and packages.
%% See http://texdoc.net/pkg/fontawesome5 for full list of symbols.

% Change the page layout if you need to
\geometry{left=1.25cm,right=1.25cm,top=1.5cm,bottom=1.5cm,columnsep=1.0cm}

% The paracol package lets you typeset columns of text in parallel
\usepackage{paracol}

% Change the font if you want to, depending on whether
% you're using pdflatex or xelatex/lualatex
% WHEN COMPILING WITH XELATEX PLEASE USE
% xelatex -shell-escape -output-driver="xdvipdfmx -z 0" sample.tex
\ifxetexorluatex
  % If using xelatex or lualatex:
  \setmainfont{Roboto Slab}
  \setsansfont{Lato}
  \renewcommand{\familydefault}{\sfdefault}
\else
  % If not using pdflatex:
  \usepackage[rm]{roboto}
  \usepackage[defaultsans]{lato}
  \usepackage{sourcesanspro}
  \renewcommand{\familydefault}{\sfdefault}
\fi

% Change the colours if you want to
\definecolor{SlateGrey}{HTML}{2E2E2E}%1
\definecolor{DarkPastelRed}{HTML}{450808}%2
\definecolor{PastelRed}{HTML}{8F0D0D}%3
\definecolor{GoldenEarth}{HTML}{E7D192}%4
\definecolor{color1}{HTML}{003545}
\definecolor{color2}{HTML}{00454A}
\definecolor{color3}{HTML}{3C6562}
\definecolor{color4}{HTML}{ED6363}
\colorlet{name}{black}
\colorlet{tagline}{color3}
\colorlet{heading}{color2}
\colorlet{headingrule}{color4}
\colorlet{subheading}{color3}
\colorlet{accent}{color3}
\colorlet{emphasis}{color1}
\colorlet{body}{black}


% Change some fonts, if necessary
\renewcommand{\namefont}{\Huge\rmfamily\bfseries}
\renewcommand{\personalinfofont}{\footnotesize}
\renewcommand{\cvsectionfont}{\LARGE\rmfamily\bfseries}
\renewcommand{\cvsubsectionfont}{\large\bfseries}


% Change the bullets for itemize and rating marker
% for \cvskill if you want to
\renewcommand{\cvItemMarker}{{\small\textbullet}}
\renewcommand{\cvRatingMarker}{\faCircle}
% ...and the markers for the date/location for \cvevent
\renewcommand{\cvDateMarker}{\faCalendar*[regular]}
%\renewcommand{\cvLocationMarker}{\faMapMarker}

%{title}{company or institution}{time period}{location}%
% If your CV/résumé is in a language other than English,
% then you probably want to change these so that when you
% copy-paste from the PDF or run pdftotext, the location
% and date marker icons for \cvevent will paste as correct
% translations. For example Spanish:
% \renewcommand{\locationname}{Ubicación}
% \renewcommand{\datename}{Fecha}


%% Use (and optionally edit if necessary) this .tex if you
%% want to use an author-year reference style like APA(6)
%% for your publication list
% \input{pubs-authoryear.tex}

%% Use (and optionally edit if necessary) this .tex if you
%% want an originally numerical reference style like IEEE
%% for your publication list
\usepackage[backend=biber,style=ieee,sorting=ydnt,defernumbers=true]{biblatex}
%% For removing numbering entirely when using a numeric style
\setlength{\bibhang}{1.25em}
\DeclareFieldFormat{labelnumberwidth}{\makebox[\bibhang][l]{\itemmarker}}
\setlength{\biblabelsep}{0pt}
\defbibheading{pubtype}{\cvsubsection{#1}}
\renewcommand{\bibsetup}{\vspace*{-\baselineskip}}
\AtEveryBibitem{%
  \iffieldundef{doi}{}{\clearfield{url}}%
}


%% sample.bib contains your publications
\addbibresource{my_pubs.bib}

\begin{document}
\name{Claudio A. Parra M.}
\tagline{Python Developer | Telecommunications Engineer}
%% You can add multiple photos on the left or right
% \photoR{2.8cm}{alta_cv/Globe_High}
% \photoL{2.5cm}{Yacht_High,Suitcase_High}
\headline{Telecommunications Engineer and Python developer with passion for solving complex problems and coding.}
\personalinfo{%
  % Not all of these are required!
  \email{clmaldonadop@gmail.com}
  \location{Castro, Region de Los Lagos, Chile}
  % \homepage{www.homepage.com}
  \linkedin{claudiopm}
  \github{claudioparram}
  %\orcid{0000-0000-0000-0000}
  %% You can add your own arbitrary detail with
  %% \printinfo{symbol}{detail}[optional hyperlink prefix]
  % \printinfo{\faPaw}{Hey ho!}[https://example.com/]
  \NewInfoField{wsp}{\faWhatsapp}[https://wa.me/]
  \wsp{+56962034487}
  %% Or you can declare your own field with
  %% \NewInfoFiled{fieldname}{symbol}[optional hyperlink prefix] and use it:
  % \NewInfoField{gitlab}{\faGitlab}[https://gitlab.com/]
  % \gitlab{your_id}
  %%
  %% For services and platforms like Mastodon where there isn't a
  %% straightforward relation between the user ID/nickname and the hyperlink,
  %% you can use \printinfo directly e.g.
  % \printinfo{\faMastodon}{@username@instace}[https://instance.url/@username]
  %% But if you absolutely want to create new dedicated info fields for
  %% such platforms, then use \NewInfoField* with a star:
  % \NewInfoField*{mastodon}{\faMastodon}
  %% then you can use \mastodon, with TWO arguments where the 2nd argument is
  %% the full hyperlink.
  % \mastodon{@username@instance}{https://instance.url/@username}
}

\makecvheader
%% Depending on your tastes, you may want to make fonts of itemize environments slightly smaller
\AtBeginEnvironment{itemize}{\small}

%% Set the left/right column width ratio to 6:4.
\columnratio{0.65}

% Start a 2-column paracol. Both the left and right columns will automatically
% break across pages if things get too long.
\begin{paracol}{2}

\cvsection{\faLaptop Courses and Certificates}
% Awesome Source CV LaTeX Template
%
% This template has been downloaded from:
% https://github.com/darwiin/awesome-neue-latex-cv
%
% Author:
% Christophe Roger
%
% Template license:
% CC BY-SA 4.0 (https://creativecommons.org/licenses/by-sa/4.0/)

%Section: Scholarships and additional info
\sectionTitle{Courses and Cerificates}{\faLaptop}

\begin{scholarship}
	\scholarshipentry{2020}{Tools for Database Modeling and SQL Queries, {\bf Pontificia Universidad Católica
	de Chile}}
	\scholarshipentry{2020}{Advanced Python programming tools for data processing, {\bf Pontificia Universidad
					Católica de Chile}}
	\scholarshipentry{2019}{Python programming tools for data processing, {\bf Pontificia Universidad Católica de
					Chile}}
	\scholarshipentry{2019}{Certified Ethical Hacker, {\bf EC Council, Certificate: ECC74026183859}}
	\scholarshipentry{2019}{SOAR Engineer, Security Analyst, Administrator, {\bf Demisto}}
	\scholarshipentry{2016}{Accredited Configuration Engineer. {\bf Palo Alto Networks}}
	
\end{scholarship}


\cvsection{\faSuitcase Professional experience}
% Awesome Source CV LaTeX Template
%
% This template has been downloaded from:
% https://github.com/darwiin/awesome-neue-latex-cv
%
% Author:
% Christophe Roger
%
% Template license:
% CC BY-SA 4.0 (https://creativecommons.org/licenses/by-sa/4.0/)

%Section: Work Experience at the top
\sectionTitle{Professional Experience}{\faSuitcase}
%\renewcommand{\labelitemi}{$\bullet$}
\begin{experiences}
  \experience
    {January 2022}   {Python Developer}{Tata Consultancy Services}{Santiago (Remote)}
    {Today} 
    {
      \begin{itemize}
        \item Contractor Developer in a financial company.
        \item Develop new features for the client's Tool.
        \item Improvements to the client's tool.
        \item Design, Schedule and Develop of the new products
        \item Support for internal users.
      \end{itemize}
    }{Python,Fast API, Flask, Docker, Git, Visual Studio Code, Jenkins, Nexus, Linux, Google Cloud, Airflow}
  \emptySeparator
  \experience
    {July 2019} {Development Engineer}{Satelnet SpA.}{Puerto Varas}
    {January 2022}    
    {
      \begin{itemize}
        \item Develop of IIOT{\it (Industrial Intenet of Things)} Solutions.
        \item Develop tools for data acquisition and storage.
        \item Design of software architecture and hardware for the solutions.
        \item Protocol development for LoRa based communications.
        \item Continuous improvements for the active services.                           
      \end{itemize}
    }{Python, Flask, Visual Studio Code, Linux, MongoDB, MariaDB, Google Cloud}
  \emptySeparator
  \experience
    {September 2018}     {Cybersecurity Engineer}{NeoSecure S.A}{Santiago}
    {July 2019}    
    {
      \begin{itemize}
        \item QRadar based Threat Hunting.
        \item Investigation of new threats and dynamic analysis in Sandbox.              
        \item Support for incident response.                   
        \item Writing of notices and security alerts.                          
        \item SOAR Demisto and EDR Crowdstrike Specialist.                      
        \item Development of company's software integrations for Demisto                                
      \end{itemize}
    }{CrowdStrike, Demisto, SOAR, QRadar, Python, Visual Studio Code}
  \emptySeparator
  \experience
  {January 2018}{Level 2 Security Analyst}{Neosecure S.A}{Santiago}
  {September 2018}{
    \begin{itemize}
      \item Static malware analysis.
      \item requirements attention related to investigation and analysis.
      \item ArcSight SIEM based Threat Hunting.
      \item Automation of some tasks and reports from Level 1 and 2 of Analysts.
    \end{itemize}
  }{ArcSight, Python, Linux}
  \emptySeparator
  \experience
  {July 2016}{Level 1 Security Analyst}{Neosecure S.A}{Santiago}
  {December 2017}{
    \begin{itemize}
      \item Security and availability monitoring with ArcSight.
      \item Availability monitoring based on Nagios.
      \item Alert checking in Linux platforms.
      \item Reports delivery for clients.
      \item Automation of some tasks related to Level 1 of Analysts.
    \end{itemize}
  }{ArcSight, Python, Linux, Nagios, VisualBasic}
  \emptySeparator
  \experience
  {March 2014}{Assistant Student: ICT Extension}{Universidad de Concepción}{Concepción}
  {December 2015}{
    \begin{itemize}
      \item Community Manager of Career's social media.
      \item Support for career's events organization.
      \item Posts in the carrer's web page.
    \end{itemize}
  }{WordPress, HTML}
  \emptySeparator
  \experience
  {March 2014}{Assistant Student}{Universidad de Concepción}{Concepción}
  {December 2015}{
    \begin{itemize}
      \item Assistant in Applied Statistics and Random Processes courses
      \item Laboratory Assistant in the course of Data Networks Laboratory.  
      \item Practical classes teacher (practical guides resolution).
      \item Development of practical guides and evaluations.
    \end{itemize}
  }{Python, Linux, MATLAB, \LaTeX}
\end{experiences}


\switchcolumn

\cvsection{\faGraduationCap Education}
% Awesome Source CV LaTeX Template
%
% This template has been downloaded from:
% https://github.com/darwiin/awesome-neue-latex-cv
%
% Author:
% Christophe Roger
%
% Template license:
% CC BY-SA 4.0 (https://creativecommons.org/licenses/by-sa/4.0/)

%Section: Scholarships and additional info
\sectionTitle{Academical Formation}{\faGraduationCap}

\begin{scholarship}
	\scholarshipentry{2021}
					{Diplomma in Data Science, Universidad del Desarrollo}
	\scholarshipentry{2021}
					{Diplomma in Data Analysis, Universidad del Desarrollo}
	\scholarshipentry{2015}
					{Telecommunications engineer, Universidad de Concepción}
\end{scholarship}

\cvsection{\faLanguage Languages}

\cvskill{Spanish}{5}
\cvskill{English}{3.5}

\cvsection{\faKeyboard Skills}
\cvskill{Python, Bash}{4.5}
\cvskill{Docker, Git}{4}
\cvskill{SQL, NoSQL}{4}
\cvskill{VS Code}{4}
\cvskill{Agile}{4.5}
\cvskill{Software Engineering}{4}

\divider

Coding: \tagenv{C/C++,JS}

\divider

Frameworks: \tagenv{\textbf{FastAPI}, Flask}

\divider

DB: \tagenv{MySQL, SQLite, BigQuery}

\divider

OS: \tagenv{\textbf{Linux}, macOS, Windows}

\divider

Tools: \tagenv{GCP, Airflow, GitHub}

\divider

Mgmt: \tagenv{Agile, Scrum, Attlasian}\\

%% Supports X.5 values.

%% Yeah I didn't spend too much time making all the
%% spacing consistent... sorry. Use \smallskip, \medskip,
%% \bigskip, \vspace etc to make adjustments.

% \divider

\cvsection{\faFile Publications}
\mynames{Aqueveque/C.\bibnamedelimi Maldonado}

\nocite{*}

\printbibliography[heading=pubtype,title={\printinfo{\faFile*[regular]}{Publicaciones en Revistas}},type=article]

%\cvsection{Referees}

%% Awesome Source CV LaTeX Template
%
% This template has been downloaded from:
% https://github.com/darwiin/awesome-neue-latex-cv
%
% Author:
% Christophe Roger
%
% Template license:
% CC BY-SA 4.0 (https://creativecommons.org/licenses/by-sa/4.0/)

%Section: Références
% \sectionTitle{References}{\faQuoteLeft}

% \begin{referees}
% 	\referee
% 	{Eduardo Pichardo P.}
% 	{Squad Lead}
% 	{Equifax}
% 	{e.pichardo@equifax.com}
% 	{+56 9 3308 5267}
% 	\referee
% 	{Brian Uribe M.}
% 	{Solutions Engineer}
% 	{Akamai Technologies}
% 	{buribe@akamai.com}
% 	{+56 9 8918 4003}
% \end{referees}

% \begin{referees}
% 	\referee
% 	{Braulio Torres S.}
% 	{Leader Specialist}
% 	{NTT Data}
% 	{btorrsan@nttdata.com}
% 	{+56 9 9513 0977}
% 	\referee
% 	{Carlos Cuevas B.}
% 	{Managed Services Manager}
% 	{ISecurity}
% 	{ccuevas@isecurityqa.com}
% 	{+56 9 9221 0445}
% \end{referees}


\cvref{Eduardo Pichardo P.}{Squad Lead - Equifax}{e.pichardo@equifax.com}{+56 9 3308 5267}

\divider

\cvref{Braulio Torres S.}{I+D Team - Satelnet}{btorrsan@nttdata.com}{+56 9 9513 0977}

\divider

\cvref{Brian Uribe M.}{Analysts Team Leader - Neosecure}{buribe@akamai.com}{+56 9 8918 4003}

\divider

\cvref{Carlos Cuevas B.}{Cyber Int. Team Leader - Neosecure}{ccuevas@isecurityqa.com}{+56 9 9221 0445}

\end{paracol}

\end{document}